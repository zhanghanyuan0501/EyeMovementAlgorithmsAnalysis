\documentclass[12pt,a4paper,twoside,openright,titlepage]{report}
\usepackage{geometry}
    \geometry{
        a4paper,
        top=25mm,
        bottom=25mm,
        left=30mm,
        right=25mm
    }
\usepackage{float}
\usepackage{blindtext}
\usepackage[T1]{fontenc}
\usepackage{polski}
\usepackage{graphicx}
\usepackage{subfiles}

\title{Analiza wybranych algorytmów wykrywania składowych ruchu oka}
\author{inż. Paweł Kucia}
\begin{document}
    \newgeometry{centering}
        \begin{titlepage}
            \begin{figure}[h!]
                \centering
                \includegraphics[width=200pt]{resources/polslherb.JPG}
            \end{figure}
            \textsf{
                \begin{center}
                    \Large{
                        Politechnika Śląska\\
                        Wydział Automatyki, Elektroniki i Informatyki\\
                        Kierunek Informatyka\\[1.5\baselineskip]
                    }
                    \LARGE{
                        \textbf{Praca dyplomowa magisterska\\[1.5\baselineskip]}
                    }
                    \Large{
                        Analiza wybranych algorytmów wykrywania składowych ruchu oka\\[1.5\baselineskip]
                    }
                \end{center}
                \begin{flushright}
                    Autor pracy: inż. Paweł Kucia\\
                    Promotor: dr inż. Katarzyna Harężlak\\
                \end{flushright}
                \begin{center}
                    \mbox{}
                    \vfill
                    Gliwice, październik 2019
                \end{center}
            }
        \end{titlepage}
    \restoregeometry
    \begin{large}
        \tableofcontents
        \let\cleardoublepage\clearpage
        \chapter{Wstęp}
            \subfile{ch1}
        \chapter{Analiza dziedziny przedmiotowej}
            \subfile{ch2}
        \chapter{Opis projektu}
            \section{Wykorzystane narzędzia}
        \chapter{Specyfikacja zewnętrzna}
            \blindtext
        \chapter{Specyfikacja wewnętrzna}
            \blindtext
        \chapter{Badania}
            \blindtext
            \section{Przeprowadzone badania}
            \section{Wyniki badań}
        \chapter{Podsumowanie i wnioski}
            \blindtext
        \addcontentsline{toc}{chapter}{Bibliografia i odwołania}
        \begin{thebibliography}{9}
            \bibitem{lamport94}
                Leslie Lamport (1994),
                \emph{\LaTeX: A Document Preparation System}.
                Addison Wesley, Massachusetts,
                2nd Edition,
            \bibitem{GazeEyeTrackingSolutions}
                Maria Laura Mele, Stefano Frederici (2012)
                \emph{Gaze and eye-tracking solutions for psychological research},
                Cogn Process (2012) 13 (Suppl 1):S261–S265,
            \bibitem{Main}
                Dario D. Salvucci, Joseph H. Goldberg (2000)
                \emph{Identifying Fixations and Saccades in Eye-Tracking Protocols},
                "Eye Tracking Research and Applications Symposium 2000"
            \bibitem{Python}
                Mark Lutz (2009)
                \emph{Python. Wprowadzenie. Wydanie IV},
                HELION S.A.
            \bibitem{CarSteering}
                M. F. Land, D. N. Lee (1994),
                \emph{Where we look when we steer},
                Nature 369(6483):742-744
            \bibitem{Advertising}
                Jian Mou (2017),
                Effects of social popularity and time scarcity on online consumer behaviour regarding smart healthcare products: An eye-tracking approach. \emph{Computers in Human Behavior, 78, 74–89.},
                https://doi.org/10.1016/j.chb.2017.08.049
        \end{thebibliography}
    \end{large}
\end{document}