\documentclass[14pt,a4paper,twoside,openright,titlepage]{extbook}

\frontmatter
\usepackage{geometry}
    \geometry{
        a4paper,
        top=25mm,
        bottom=25mm,
        left=30mm,
        right=25mm,
        headheight=17pt
    }
\usepackage{float}
\usepackage{blindtext}
\usepackage[T1]{fontenc}
\usepackage{hyperref}
\usepackage{polski}
\usepackage{graphicx}
\usepackage{subfiles}
\usepackage{fancyhdr}
\usepackage{nopageno}
 
\newenvironment{tttabular}[1]%
{\ttfamily \small \begin{tabular}{#1}}%
{\end{tabular}}
\hypersetup{linktoc=section}
\linespread{1.3}
\setlength{\parindent}{0pt}
\setlength{\parskip}{1ex plus 0.5ex minus 0.2ex}  
\title{Analiza wybranych algorytmów wykrywania składowych ruchu oka}
\author{inż. Paweł Kucia}
\begin{document}
    \newgeometry{centering}
        \subfile{titlepage}
        \cleardoublepage
        \subfile{oswiadczenia}
        \cleardoublepage
    \restoregeometry
    
    \mainmatter
    \linespread{1.0}\tableofcontents
    \let\cleardoublepage\clearpage
    \pagestyle{fancy}
    \fancypagestyle{plain}{}
    \fancyhf{}
    \fancyhead[EL,OR]{\thepage}
    \fancyhead[ER]{\textit{ \nouppercase{\leftmark}}}
    \fancyhead[LO]{\textit{ \nouppercase{\rightmark}}}
    \renewcommand{\headrulewidth}{0.4pt}

    \chapter{Wstęp}
        \subfile{ch1}
    \chapter{Analiza dziedziny przedmiotowej}
        \subfile{ch2}
    \chapter{Opis projektu}
        \subfile{ch3}
    \chapter{Specyfikacja aplikacji badającej}
        Ten rozdział ma za zadanie zaprezentowanie sposobu działania aplikacji służącej do przeprowadzania badań.
        \section{Specyfikacja zewnętrzna}
        \section{Specyfikacja wewnętrzna}
    \chapter{Badania}
        \blindtext
        \section{Przeprowadzone badania}
        \section{Wyniki badań}
    \chapter{Podsumowanie i wnioski}
        \blindtext
    \backmatter
    \fancyhf{}
    \pagestyle{plain}
    \fancypagestyle{plain}{}
    \renewcommand{\headrulewidth}{0pt}
    \begin{thebibliography}{99}
        \bibitem{lamport94}
            Leslie Lamport (1994),
            \emph{\LaTeX: A Document Preparation System}.
            Addison Wesley, Massachusetts,
            2nd Edition,
        \bibitem{GazeEyeTrackingSolutions}
            Maria Laura Mele, Stefano Frederici (2012)
            \emph{Gaze and eye-tracking solutions for psychological research},
            Cogn Process (2012) 13 (Suppl 1):S261–S265,
        \bibitem{Main}
            Dario D. Salvucci, Joseph H. Goldberg (2000)
            \emph{Identifying Fixations and Saccades in Eye-Tracking Protocols},
            "Eye Tracking Research and Applications Symposium 2000"
        \bibitem{Python}
            Mark Lutz (2009)
            \emph{Python. Wprowadzenie. Wydanie IV},
            HELION S.A.
        \bibitem{CarSteering}
            M. F. Land, D. N. Lee (1994),
            \emph{Where we look when we steer},
            Nature 369(6483):742-744
        \bibitem{Advertising}
            Jian Mou (2017),
            Effects of social popularity and time scarcity on online consumer behaviour regarding smart healthcare products: An eye-tracking approach. \emph{Computers in Human Behavior, 78, 74–89.},
            https://doi.org/10.1016/j.chb.2017.08.049
    \end{thebibliography}
    \part*{Dodatki}
        \section*{Zawartość płyty}
            Do pracy dołączona została płyta z następującą zawartością:
            \begin{itemize}
                \item asd
            \end{itemize}
        \listoffigures
        \listoftables
\end{document}