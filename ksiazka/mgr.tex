\documentclass[14pt,a4paper,twoside,openright,titlepage]{extbook}

\frontmatter
\usepackage{geometry}
    \geometry{
        a4paper,
        top=25mm,
        bottom=25mm,
        left=30mm,
        right=25mm,
        headheight=17pt
    }
\usepackage{float}
\usepackage{blindtext}
\usepackage[T1]{fontenc}
\usepackage{hyperref}
\usepackage{polski}
\usepackage{graphicx}
\usepackage{subfiles}
\usepackage{fancyhdr}
\usepackage{nopageno}
\usepackage{caption}
\usepackage{subcaption}
\usepackage[linesnumbered,figure,noalgohanging]{algorithm2e}
\usepackage{setspace}
\usepackage{listings}
\usepackage[normalem]{ulem}
\usepackage{todonotes}
%Config latex
\lstset{
  columns=fullflexible,
  frame=single,
  breaklines=true,
  numbers=left,
  captionpos=b,
  basicstyle=\small
}
\hypersetup{linktoc=section}
\linespread{1.3}
\setlength{\parindent}{0pt}
\setlength{\parskip}{1ex plus 0.5ex minus 0.2ex}  
\renewcommand*{\lstlistlistingname}{Spis fragmentów kodu}
\renewcommand*{\lstlistingname}{Kod}

\title{Analiza wybranych algorytmów wykrywania składowych ruchu oka}
\author{inż. Paweł Kucia}
%Koniec konfig latex
\begin{document}
    \newgeometry{centering}
        \subfile{titlepage}
        \cleardoublepage
        \subfile{oswiadczenia}
        \cleardoublepage
    \restoregeometry
    
    \mainmatter
    \linespread{1.0}\tableofcontents
    \let\cleardoublepage\clearpage
    \pagestyle{fancy}
    \fancypagestyle{plain}{}
    \fancyhf{}
    \fancyhead[EL,OR]{\thepage}
    \fancyhead[ER]{\textit{ \nouppercase{\leftmark}}}
    \fancyhead[LO]{\textit{ \nouppercase{\rightmark}}}
    \renewcommand{\headrulewidth}{0.4pt}

    \chapter{Wstęp}
        \subfile{ch1}
    \chapter{Analiza dziedziny przedmiotowej}
        \subfile{ch2}
    \chapter{Opis projektu badawczego}
        \subfile{ch3}
    \chapter{Badania}
        \subfile{ch4}
    \chapter{Podsumowanie i wnioski}
        \blindtext
    \backmatter
    \fancyhf{}
    \pagestyle{plain}
    \fancypagestyle{plain}{}
    \renewcommand{\headrulewidth}{0pt}
    \begin{thebibliography}{99}
        \bibitem{Main}
            Dario D. Salvucci, Joseph H. Goldberg (2000),
            \emph{Identifying Fixations and Saccades in Eye-Tracking Protocols},
            Eye Tracking Research and Applications Symposium 2000
        \bibitem{MLPython}
            pod red. Fabian Pedregosa, Gael Varoquaux, Alexandre Gramfort, Vincent Michel, Bertrand Thirion, itd.
            \emph{Scikit-learn: Machine Learning in Python},
            Journal of Machine Learning Research 12 (2011) 2825-2830
        \bibitem{EvaluationMethodology}
            Gustav Larsson,
            \emph{Evaluation Methodology of Eye
            Movement Classification Algorithms},
            Królewski Instytut Technologiczny w Sztokholmie
        \bibitem{Advertising}
            Jian Mou (2017),
            Effects of social popularity and time scarcity on online consumer behaviour regarding smart healthcare products: An eye-tracking approach. \emph{Computers in Human Behavior, 78, 74–89.},
            \url{https://doi.org/10.1016/j.chb.2017.08.049}
        \bibitem{MongoDB}
            Karl Seguin (2011),
            \emph{The Little MongoDB Book},\\
            \url{https://openmymind.net/mongodb.pdf}
        \bibitem{Latex}
            Leslie Lamport (1994),
            \emph{\LaTeX: A Document Preparation System}.
            Addison Wesley, Massachusetts,
            2nd Edition,
        \bibitem{GazeEyeTrackingSolutions}
            Maria Laura Mele, Stefano Frederici (2012)
            \emph{Gaze and eye-tracking solutions for psychological research},
            Cogn Process (2012) 13 (Suppl 1):S261–S265,
        \bibitem{Python}
            Mark Lutz (2009)
            \emph{Python. Wprowadzenie. Wydanie IV},
            HELION S.A.
        \bibitem{OtherEyetrack}
            Michael Raschke, Tanja Blascheck, Michael Burch (2013),
            \emph{Visual Analysis of Eye Tracking Data},
            Institute for Visualization and Interactive Systems, University of Stuttgart, Germany
        \bibitem{CarSteering}
            M. F. Land, D. N. Lee (1994),
            \emph{Where we look when we steer},
            Nature 369(6483):742-744
        \bibitem{MachineLearning}
            Raimondas Zemblys, Diederick C. Niehorster, Oleg Komogortsev, Kenneth Holmqvist,
            \emph{Using machine learning to detect events in eye-tracking data},
            Psychonomic Society, Inc. 2017
        \bibitem{metodyeyetrack}
            Robert Gabriel Lupu*, Florina Ungureanu (2013),
            \emph{A Survey Of Eye Tracking Methods And Applications},
            “Gheorghe Asachi” Technical University of Iaşi,
            Rozdział 2
    \end{thebibliography}
    \part*{Dodatki}
        \section*{Zawartość płyty}
            Do pracy dołączona została płyta z następującą zawartością: oryginał pracy magisterskiej, pliki źródłowe aplikacji, prezentacja, praca magisterska posiadająca tylko tekst.
        \listoffigures
        \listoftables
        \lstlistoflistings
\end{document}