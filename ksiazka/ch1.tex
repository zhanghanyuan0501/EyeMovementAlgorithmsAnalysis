Technologia eye-trackingu \textit{(pol. okulografia)} na przestrzeni ostatnich lat stała się powszechnie stosowaną w różnorakich dziedzinach nauki, takich jak: neurochirurgia, psychologia eksperymentalna, informatyka. Pozwala ona na wykorzystanie prześledzonego ruchu gałek ocznych, zazwyczaj za pomocą specjalnie stworzonego w tym celu urządzenia, i wykorzystanie go w poniższych dziedzinach. Rozwój tej dziedziny umożliwia poznanie procesów zachodzących w gałce ocznej, celem użycia ich w powyżej przedstawionych dziedzinach. W tym celu tworzone są różne fizyczne urządzenia pozwalające na przetworzenie wykrytych fiksacji i sakad ruchu gałek ocznych do plików znajdujących się na komputerach, co pozwala przeprowadzać dalszą analizę.\\
Poniższa praca ma na celu zaprezentowanie algorytmów wykrywania fiksacji z danych pobranych za pomocą urządzenia fizycznego, jak również porównania wybranych algorytmów, oraz omówienia wyników przeprowadzonych badań.\\
Opis wykonanej pracy został podzielony na kilka rozdziałów. W następnym rozdziale zaprezentowano dziedzinę przedmiotową oraz opisane zostały algorytmy wykorzystane w pracy. W rozdziale trzecim przedstawiono wykorzystane narzędzia oraz moduły w aplikacji implementującej algorytmy. Rozdziały czwarty oraz piąty podejmują tematykę odpowiednio specyfikacji zewnętrznej oraz wewnętrznej aplikacji. Kolejny rozdział prezentuje sposób przeprowadzenia oraz wyniki badań. Ostatni rozdział pracy podsumowuje wykonaną pracę oraz zawiera wnioski z przeprowadzonej pracy.