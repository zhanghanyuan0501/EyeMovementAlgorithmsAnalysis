\label{chapter:ch1}
Dziedzina okulografii \textit{(z ang. eye-tracking)} na przestrzeni ostatnich kilkudziesięciu lat bardzo się rozpowszechniła. Jej zastosowanie możemy znaleźć w wielu aspektach życia oraz nauki, nawet nie zdając sobie z tego sprawy. Wraz z jej rozwojem umożliwiane jest coraz dokładniejsze prześledzenie ruchu gałek ocznych badanego podmiotu, co pozwala na dalszą analizę tych danych. Wynikiem takiej analizy może być na przykład mapa cieplna zmierzonego ruchu oka na banerze reklamowym, co w rezultacie dostarcza reklamodawcy kolejne informacje dotyczące sposobu tworzenia reklam, by rozszerzyć zasięg osób, do których one trafiają.\par
Wraz z rozwojem okulografii, wzrosło zapotrzebowanie na co raz dokładniejsze urządzenia służące do wykonywania dokładnych pomiarów ruchu, takich jak kamery, elektrody. W większości przypadków te dane są analizowane pod kątem wykrywania fiksacji lub sakad ruchu oka.\par
Celem poniższej pracy jest zaprezentowanie analizy wybranych algorytmów wykrywania fiksacji oraz sposobu ich implementacji. Analiza jest wykonywana pod kątem zużycia pamięciowego, czasu trwania algorytmów, liczby wykrytych fiksacji. Badany również będzie wpływ parametrów zewnętrznych na poprawność i dokładność algorytmów. W pracy zaprezentowano również jeden przykład rozwiązania bazującego na coraz to popularniejszej technologii uczenia maszynowego. W tym algorytmie będzie można zbadać dokładność wyników końcowych względem tradycyjnych algorytmów.\par
Opis wykonanej pracy został podzielony na kilka rozdziałów. Drugi rozdział pracy przedstawia opis teoretyczny dziedziny okulografii oraz związanej z nią tematyki budowy ludzkiej gałki ocznej. W tym rozdziale scharakteryzowano również metody prezentacji danych a także opisano różne algorytmy wykrywania fiksacji, w tym te wybrane do dalszej analizy. Kolejny rozdział stanowi opis stworzonego projektu badawczego. Ten rozdział rozpoczęto od charakterystyki utworzonej aplikacji, wraz z opisem wykorzystanych narzędzi. Po tym zaprezentowano specyfikację zewnętrzną aplikacji, w tym parametry uruchomieniowe dla programu, format danych wejściowych oraz wyjściowych. Pod koniec tej sekcji przedstawiono metodę wyświetlania danych końcowych. Kolejnym podrozdziałem jest opis specyfikacji wewnętrznej programu. Rozpoczyna się on od zaprezentowania wymagań aplikacji, czyli koniecznych modułów oraz środowisk, celem poprawnego jej uruchomienia. Kolejnymi zademonstrowanymi funkcjonalnościami są: metodyka kalibracji danych wejściowych, implementacja obsługi bazy danych, opis przygotowania danych do analizy za pomocą wybranych algorytmów. Sekcje \ref{ssec:implementidt}, \ref{ssec:implementivt}, \ref{ssec:machinelearningalg} prezentują wdrożone algorytmy wykrywania fiksacji. Ostatnie 3 sekcje opisują w jaki sposób zaprezentowano wyniki, jak wykonano pomiary czasu oraz  przedstawiono bibliotekę dotyczącą analizy wykorzystania pamięci w języku Python 3.\\
Głównym zadaniem czwartego rozdziału jest przedstawienie wyników algorytmów oraz sposobu ich analizy. Pierwsze dwa podrozdziały opisują parametry maszyny, na której przeprowadzano badania oraz parametry związane ze stanowiskiem pomiarowym, danymi pomiarowymi. W sekcji \ref{ssec:calibration} zaprezentowano sposoby implementacji regresji dla kalibracji danych wejściowych, jak również otrzymane skalibrowane punkty. Następna sekcja została poświęcona sprawdzeniu wpływu parametrów wejściowych algorytmów na wyniki końcowe. Analiza ta odbywa się za pomocą porównania czasów trwania algorytmów, sprawdzenia wykorzystania pamięci przez algorytmy oraz analizę liczby otrzymanych fiksacji. Ostatni fragment w tym rozdziale przyrównuje te wyniki do siebie, oraz przeprowadza dokładniejszą analizę tych wyników. Ostatni rozdział przeznaczono na podsumowanie oraz wnioski z przeprowadzonej analizy.