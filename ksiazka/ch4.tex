W tym rozdziale przedstawiono przeprowadzone badania, sposób ich przeprowadzenia, przedstawiono sprzęt, na którym przeprowadzono badania.
\section{Maszyna do badań}
Wszystkie testy oraz badania zostały przeprowadzone na maszynie zaprezentowanej w tabeli \ref{tab:machine}. W tabeli wymieniono tylko znaczące elementy, tzn. procesor, pamięć RAM oraz jej szybkość, dysk twardy, jego prędkość obrotową oraz typ. Wszystkie te podzespoły mogą wpływać na czas trwania elementów działania aplikacji. Ze względu na jednowątkowość aplikacji, nie podaje się ilości wątków.
\begin{table}[H]
    \centering
    \begin{tabular}{|l|l|}
    \hline
    Procesor       & Intel Core-i5 8600 @ 3.10 GHz          \\ \hline
    Pamięć RAM     & 16GB DDR4 @ 3200 MHz \\ \hline
    Typ dysku twardego           & HDD              \\ \hline
    Dysk twardy         & Toshiba HDWD110              \\ \hline
    Prędkość obrotowa         & 7200 obr./min              \\ \hline
    \end{tabular}
    \caption{Maszyna do badań}
    \label{tab:machine}
\end{table}
\section{Parametry stanowiska pomiarowego}
30 pomiarów
\section{Przeprowadzone badania}
\subsection{Wybór metodyki regresji}
\label{ssec:regression}
\subsection{Analiza wpływu parametrów algorytmów}
\label{ssec:queryparameters}
\subsection{Porównanie czasów trwania algorytmów}
\label{ssec:times}
\subsection{Analiza obciążenia pamięci}
\label{ssec:memory}
\subsection{Wpływ typu danych wejściowych}
\label{ssec:entrydata}
\subsection{Dokładność algorytmów i porównanie punktów wynikowych}