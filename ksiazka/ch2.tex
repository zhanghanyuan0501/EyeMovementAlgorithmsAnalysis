W poniższym rozdziale zawarty został krótki opis oka, analiza teoretyczna dziedziny eye-trackingu, jak również pokaznie algorytmów wykrywania fiksacji wykorzystanych do przeprowadzenia badań. Rozdział ten oparto głównie o prace \cite{Main}, \cite{EvaluationMethodology}, \cite{MachineLearning}.
\section{Opis ludzkiej gałki ocznej}
Zadaniem poniższej sekcji jest zaprezentowanie podstawowych terminologii związanych z ludzkim okiem oraz jego budową. Krótka analiza oka została przygotowana zgodnie z pracą numer \cite{EvaluationMethodology}.\par
Jednym z najważniejszych organów w ludzkim ciele jest oko. Służy ono do dostarczania większości informacji dotyczących otoczenia w jakim się znajdujemy. Około 10\% komórek mózgowych jest zaangażowanych przy interpretacji oraz analizie sygnałów dostarczanych z tego narządu.\par
Działanie oka polega na projekcji światła wpadającego do rogówki, najbardziej zewnętrznej część oka, poprzez źrenicę, która może być regulowana tęczówką w zależności od ilości promieni świetlnych, przechodzi ono następnie przez soczewkę, która załamuje te promienie, ciało szkliste, natrafiając na koniec na wewnętrzną warstwę oka, czyli siatkówkę, która składa się z dwóch typów fotoreceptorów: czopki - około 6 milionów, pręciki - około 100 milionów. Na siatkówce obraz, który powstał jest obrócony. Nerw wzrokowy transmituje dalej odwrócony obraz jako sygnał nerwowy do ośrodków wzrokowych kory mózgowej. Ważnymi elementami przy poprawnie funkcjonującym oku są również ciało rzęskowe, posiadające promieniście ułożone fałdy, wydzielające wodnistą ciecz, odpowiadającą za sztywność gałki ocznej, mięsień rzęskowy, który zmienia krzywiznę soczewki, zmieniając jej ogniskową, przez co występuje zjawisko akomodacji oka. 
\begin{figure}[H]
    \centering
    \captionsetup{justification=centering,margin=2cm}
    \includegraphics[width=0.8\linewidth]{resources/oko_galka.jpg}
    \caption[Przekrój oka.]{Uproszczony schemat gałki ocznej.\\\hspace{\textwidth} 
    \small(źródło: \url{https://eszkola.pl/fizyka/schemat-budowy-oka-i-wady-wzroku-4225.html} [dostęp 01.10.2019])}
    \label{fig:budowaoka}
\end{figure}
Akomodacja oka to zjawisko pozwalające oku na dostosowanie się do oglądania przedmiotów znajdujących się na różnych odległościach. Jak wspominiano wyżej fotoreceptory znajdujące się w oku pozwalają nam rozpoznać obraz ze światła wpadającego do oka, czopki odpowiadają za widzenie fotopowe, pręciki za widzenie skotopowe. Widzenie fotopowe pozwala na rozpoznawanie kolorów (czerwony, zielony, niebieski) przy dobrym oświetleniu, które nie może być zbyt intensywne, ponieważ czopki mogą ulec przesyceniu. Widzenie skotopowe to możliwość obserwacji czarno-białego obrazu przy słabszym oświetleniu, jednak obraz nie jest tak dokładny jak przy widzeniu fotopowym. Przykład zakresu absorbcji fal świetlnych można zaobserwować na rysunku \ref{fig:czopki}.
\begin{figure}[H]
    \centering
    \captionsetup{justification=centering,margin=2cm}
    \includegraphics[width=0.9\linewidth]{resources/czopki.png}
    \caption[Względna absorbcja światła przez czopki oraz pręciki.]{Względna absorbcja światła przez czopki (K, Ś, D) oraz pręciki (Pr).\\\hspace{\textwidth}
    \small(źródło: \url{https://upload.wikimedia.org/wikipedia/commons/9/9a/Cone-response-pl.svg} [dostęp 10.09.2019])}
    \label{fig:czopki}
\end{figure}
Najważniejszym dla nas elementem siatkówki jest plamka żółta, umożliwiającą największą rozdzielczość widzenia, ze względu na najwyższe stężenie fotoreceptorów w ludzkim oku. Umożliwia ona ludziom np. czytanie tekstu. Im dalej od plamki żółtej, tym zwiększa się ilość pręcików, a przez to zmniejsza się dokładność obrazu.
\section{Metody wykrywania ruchu gałek ocznych}
\label{sec:movement}
Jednym z głównych kryteriów dla mechanizmów wykrywania ruchu gałek jest jego inwazyjność. Może ona powodować dyskomfort u badanego podmiotu, co w rezultacie może dać przekłamane wyniki, przez to że użytkownik może zachowywać się mniej naturalnie. Wiele badań również wykazuje to, iż preferowane jest, żeby badano jak największą liczbę osób z małą znajomością tematyki śledzenia ruchu oka. Ta sekcja bazuje na danych znalezionych w pracy \cite{metodyeyetrack}. Poniżej przedstawiono trzy główne metody mierzenia ruchu oka.\par
\subsection{Metody bazujące na soczewkach kontaktowych}
\label{ssec:lenses}
Ta kategoria sposobów mierzenia ruchu oka zakłada od badanego założenia na gałki oczne soczewek kontaktowych, które posiadają lustra, bądź cewkę z kablem, która obkrążała soczewkę kontaktową. Cewka ta była połączona do systemu cewek magnetycznych, co powodowało mierzenie ruchu gałek ocznych za pomocą zmian sił pola magnetycznego. Pomimo większego rozwoju technologicznego, ze względu na inwazyjność całego rozwiązania, jak również konieczność unieruchomienia głowy w trakcie przeprowadzenia pomiarów. Podgląd takich rozwiązań zaprezentowano na rysunku \ref{fig:soczewki}.
\begin{figure}[H]
    \centering
    \captionsetup{justification=centering,margin=2cm}
    \includegraphics[width=0.9\linewidth]{resources/soczewki.png}
    \caption{Przykład rozwiązań bazujący na soczewkach kontaktowych.}
    \label{fig:soczewki}
\end{figure}
\subsection{Elektrookulografia (EOG)}
\label{ssec:eog}
Poniższy typ rozwiązań bazuje na różnicy potencjałów pomiędzy elektrodami przymocowanymi blisko oczu. Ze względu na dużą ilość połączeń nerwowych w gałce ocznej, można odczuć zmianę napięcia w samej gałce, jak również wyczuwana jest zmiana pól elektrycznych podczas wykonywania ruchu. Amplituda tych zmian zależy od pozycji oka. Powody niewykorzystywania tych rozwiązań można przypisać do tych zaprezentowanych w sekcji \ref{ssec:lenses}.
\subsection{Kamery}
\label{ssec:cameras}
Ostatnia, trzecia metoda pobierania informacji wykorzystuje kamery, czy to umieszczone zdalnie, czy też montowane w okularach, jak zaprezentowano poglądowo na rysunkach \ref{fig:camerassub1} i \ref{fig:camerassub2}. Jak można zaobserwować, metody te ze względu na ich nieinwazyjność znalazły najszersze zastosowanie w dziedzinie pomiaru ruchu gałek ocznych. Ten sposób wyciągania danych wykorzystuje dwa typy obrazów: obraz z naturalnego światła oraz z podczerwieni. Obraz z naturalnego światła jest podejściem pasywnym, które pobiera dane z odbitego światła z oka, wynikiem tego obrazu jest zarys ruchu soczewki. Wadą tego rozwiązania jest zależność od źródła światła, gdyż pomiary przeprowadzane w złym świetle mogą powodować przekłamania, jak również w skrajnych przypadkach brak możliwości odczytu danych. Wykorzystanie odczytu światła podczerwieni eliminuje ten problem. Kolejną zaletą użycia podczerwieni jest zmiana wyniku, z zarysu ruchu soczewki na zarys ruchu źrenicy, poprzez wykorzystanie zjawiska odbicia światła z ekranu. W wypadku spojrzenia na element, odbicie jest białe, w przeciwnym stanie - czarne.
\begin{figure}[H]
    \centering
    \begin{subfigure}{.5\textwidth}
      \centering
      \includegraphics[width=\linewidth]{resources/camera.png}
      \caption{Stanowisko z kamerą}
      \label{fig:camerassub1}
    \end{subfigure}%
    \begin{subfigure}{.5\textwidth}
      \centering
      \includegraphics[width=\linewidth]{resources/glasses.jpg}
      \caption{Kamera w okularach}
      \label{fig:camerassub2}
    \end{subfigure}
    \caption{Przykłady zastosowania kamer}
    \label{fig:cameras}
\end{figure}
\section{Eye-tracking}
\label{sec:eyetracking}
W następującej sekcji zaprezentowana zostanie technologia eye-trackingu, oraz jej zastosowania.\\[\baselineskip]
Jak wspomniano we wstępie do pracy, rozwój okulografii w przeciągu ostatnich kilkudziesięciu lat pozwala nam przeanalizować sposób w jakim operują ludzkie procesy obserwacji oraz sposobu rozpoznawania obrazu. Analiza tych procesów umożliwia badającym na wykorzystywanie wyników badań w zastosowaniach komercyjnych, np. badanie sposobu patrzenia na jezdnię w trakcie jazdy pojazdem \cite{CarSteering}, analiza psychologiczna \cite{GazeEyeTrackingSolutions} czy też w wykorzystaniu komercyjnym \cite{Advertising}.\\[\baselineskip]
W celu przetworzenia danych z urządzenia pomiarowego, których przykłady zaprezentowano w podrozdziale \ref{sec:movement} do dalszej analizy stosuje się typowo dwie wartości: fiksacje, czyli miejsca, na których badana próbka wzroku się skupiła, oraz sakady, szybkie ruchy pomiędzy fiksacjami. Podział ten został odkryty w XIX wieku we Francji za pomocą obserwacji fizycznych, ponieważ zauważono, że ludzkie oko podczas czytania nie porusza się płynnie, a wykonuje "skoki" pomiędzy obszarami tekstu. Przykład takiego ruchu oraz punktów skupienia można zaobserwować na rysunku \ref{fig:fiksacje}.
\begin{figure}[H]
    \centering
    \captionsetup{justification=centering,margin=2cm}
    \includegraphics[width=0.8\linewidth]{resources/fixation_example.jpg}
    \caption[Przykład fiskacji i sakad na tekście.]{Przykład fiskacji i sakad na tekście.\\\hspace{\textwidth}
    \small(źródło: \url{https://upload.wikimedia.org/wikipedia/commons/e/ef/Reading_Fixations_Saccades.jpg} [dostęp 10.09.2019])}
    \label{fig:fiksacje}
\end{figure}
Analiza fiksacji między innymi poprzez translacje danych ruchu oka z urządzenia wejściowego na fiksacje, co pozwala również określić sakady na pomiarze. Daje to możliwość pozbycia się zbędnych danych z próbki, takich jak sakad, oraz pomniejszych ruchów oka, które mogły nastąpić przy niedokładnym pomiarze, czy mikroskopijnym ruchu oka. Zezwala to nam na zmniejszenie rozmiaru danych, poprzez zbijająnia rzeczywistych fiksacji do jednego, większego punktu danych. Najczęściej otrzymane wartości są wykorzystywane do metryk pomiaru typu czas fiksacji, prędkości i amplitudy sakad, jak również miary pomiędzy fiksacjami a sakadami. Jednak cytując \cite{Main} \emph{"w większości badań naukowych, dane z sakad nie stanowią aż takiej przydatności"}.\\[\baselineskip]
Wyniki algorytmów wykrywania fiksacji są wynikami typowo statystycznymi, tzn. możemy określić ile wystąpiło fiksacji, a przez to ile elementów jest sakadami, ale dalsza analiza danych należy do badającego. Stwarza to problem interpretacji danych, zgodnie z \cite{Main} \emph{"jednym ze sposobów walidacji tych algorytmów jest porównanie wynikowych fiksacji z wrażeniami wizualnymi obserwującego"}.
\section{Algorytmy wykrywania fiksacji}
\label{sec:fixations}
Wyróżniamy trzy główne typy algorytmów wykrywania fiksacji ze względu na obszar badania:
\begin{itemize}
    \item prędkościowe
    \item dyspersyjne
    \item powierzchniowe
\end{itemize} 
Algorytmem prędkościowym możemy nazwać algorytm, który analizuje punkty pod kątem różnicy prędkości pomiędzy nimi, biorąc pod uwagę, iż fiksacje posiadają niską prędkość pomiędzy swoimi punktami, a sakady wysoką. Algorytmy dyspersyjne bazują na odległościach pomiędzy punktami, zakładając, iż fiksacje posiadają małe odległości międzypunktowe. Algorytmy powierzchniowe to algorytmy, których zadaniem jest identyfikacja punktów w wybranych powierzchniach zainteresowania (AOI)\footnote{ang. area of interest}. Algorytmy tego rodzaju posiadają, w przeciwieństwie do innych algorytmów, możliwość identyfikacji zarówno nisko- jak i wysoko-poziomowej, przez to, iż parametrem algorytmów AOI może być fiksacja.\par

Algorytmy mogą także zostać podzielone ze względu na ich charakterystykę czasową, wyróżniamy dwa główne rodzaje:
\begin{itemize}
    \item czułe na czas trwania
    \item adaptujące się lokalnie
\end{itemize}
Ten podział został stworzony ze względu na to, iż fiksacje bardzo rzadko trwają mniej niż 100 ms, a regularny czas ich trwania potrafi wynosić od 200 ms do 400 ms. Implementacja adaptacji lokalnych umożliwia na dokładniejszy pomiar fiksacji.
\subsection{Wybrane algorytmy}
Celem poniższego podrozdziału jest opis teoretyczny algorytmów wykrywania fiksacji zastosowanych w pracy. Opis teoretyczny wybranych algorytmów bazuje na pracy \cite{Main} oraz \cite{EvaluationMethodology}.
\subsubsection{Algorytm I-VT}
\label{ssec:ivt}
Algorytm I-VT\footnote{z ang. Identification-Velocity Threshold} jest przykładem algorytmu z grupy prędkościowych. Jak wspomniano w podrozdziale \ref{sec:fixations} te algorytmy bazują na różnicach prędkości międzypunktowych. W większości przypadków te różnice wynoszą mniej niż 100 ms dla fiksacji, a więcej niż 300-400 ms dla sakad. Ze względu na proste wymagania algorytmu, nie jest on skomplikowany w implementacji. Pseudokod algorytmu I-VT zaprezentowano na rysunku \ref{fig:ivt}.
{
\begin{algorithm}[H]
    \SetAlgoLined
    \setstretch{1.35}
    \SetKwInOut{Input}{Input}
    \SetKwInOut{Output}{Output}
    Oblicz prędkości pomiędzy punktami dla każdego punktu w protokole\;
    Określ punkty poniżej progu jako fiksacje, a powyżej jako sakady\;
    Połącz wszystkie punkty fiksacji w grupy fiksacji, usuń wszystkie sakady\;
    Zmapuj każdą grupę fiksacji do punktu znajdującego się w środku każdej grupy\;
    \Return{Zmapowane punkty}\;
    \label{fig:ivt}
    \caption{Pseudokod algorytmu I-VT}
\end{algorithm}}
Jak można zauważyć w pseudokodzie powyżej, pierwszym krokiem algorytmu I-VT jest obliczenie prędkości między każdym punktem w badanym obszarze. Prędkość ta jest mierzona jako odległość między obecnym punktem a następnym (lub poprzednim) punktem. Następnie każdy punkt jest klasyfikowany jako fiksacja lub sakada w zależności od spełnienia warunku progu, którym w tym wypadku jest prędkość. Zgodnie z zasadami tego typu algorytmów, wszystkie elementy ponad granicą zaliczamy jako sakady, a resztę jako fiksacje. Kolejnym krokiem jest pozbycie się danych niepotrzebnych - sakad, i połączenie wszystkich sakad w grupy fiksacji. Ostatnim krokiem algorytmu jest wyznaczenie środka masy każdej grupy fiksacji, co pozwala nam ukazać fiksację w danym miejscu.\par
Według specyfikacji, ten algorytm wymaga podania jednego parametru wejściowego - progu prędkości.
\subsubsection{Algorytm I-DT}
\label{ssec:idt}
{
\begin{algorithm}[H]
    \SetKwInOut{Input}{Input}
    \SetKwInOut{Output}{Output}
    \setstretch{1.35}

    \While{istnieją punkty do zbadania}
    {
        Zainicjalizuj okno na pierwszych punktach, celem pokrycia progu czasowego.\;
        \If{dyspersja punktów w oknie $\leq$ próg}{
            Dodaj dodatkowe punkty do okna aż dyspersja $>$ próg\;
            Zanotuj fiksacje jako centroid punktów w oknie\;
            Usuń punkty w oknie z listy punktów\;
        }
        \Else{
            Usuń pierwszy punkt z listy punktów\;
        }
    }
    \Return{fiksacje}\;
    \caption{Pseudokod algorytmu I-DT}
    \label{fig:idt}
\end{algorithm}}
Następnym badanym algorytmem jest algorytm dyspersyjny I-DT\footnote{z ang. Identification Dispersion-Threshold}. W przeciwieństwie do algorytmów prędkościowych, algorytm I-VT wykorzystuje fakt, iż fiksacje ze względu na swoją małą prędkość mają tendencje do grupowania się. Algorytm I-DT identyfikuje te fiksacje za pomocą okien (grup punktów) o określonej dyspersji (\emph{maksymalna separacja}). Jak napisano w rozdziale \ref{ssec:ivt}, fiksacja trwa zazwyczaj 100 ms, co pozwala nam określić dolny próg wejściowy dla algorytmu na poziomie 100-200 ms. Pseudokod algorytmu można znaleźć w na rysunku \ref{fig:idt}.\par
Algorytm I-DT wykorzystuje ruchome okno, do którego należą punkty będące potencjalnymi fiksacjami. Zgodnie z powyższym pseudokodem, inicjalizacja okna następuje na podanym przez użytkownika progu czasowym. Następnie sprawdzana jest dyspersja pomiędzy punktami, którą można obliczyć za pomocą wzoru: $D = [max(x) - min(x)] + [max(y) - min(y)]$. Jest to wzór dla płaszczyzny dwuwymiarowej, jednak można zastosować inne wzory w wypadku zastosowania innych płaszczyzn. W wypadku gdy obliczona dyspersja przekroczyła zadany próg, nie znaleziono fiksacji, okno zostaje przesunięte na kolejny punkt. W przeciwnym wypadku okno należy zanotować jako fiksację. W tym oknie należy dodawać elementy aż obliczona dyspersja przekroczy próg dyspersji. Algorytm się kończy w momencie gdy wszystkie punkty zostaną przeanalizowane.\par
Algorytm I-DT wymaga podania dwóch parametrów wejściowych: progu dyspersji oraz progu czasowego. Obydwa progi można podać na podstawie obserwacji analizowanych danych. Próg czasowy zazwyczaj jest ustawiany na wartości 100-200 ms.
\subsubsection{Uczenie maszynowe}
\label{ssec:machinelearning}
W przeciągu ostatniej dekady można zauważyć znaczący wzrost zainteresowania technologiami powiązanymi z zagadnieniem sztucznej inteligencji, czy też uczenia maszynowego. Dzięki temu, iż dane wydobyte przez urządzenie śledzące ruch oka można łatwo sparametryzować, czy to przez podanie prędkości międzypunktowych, czy przez obliczanie dyspersji pomiędzy każdym punktem, również w dziedzinie eye-trackingu można zaobserwować wzrost wykorzystania technologii ML\footnote{Machine Learning}. Dla okulografii głównym wykorzystaniem tego typu rozwiązań jest możliwość automatycznej analizy danych wprowadzonych, bez konieczności podawania parametrów wejściowych, takich jak w tradycyjnych algorytmach zaprezentowanych w sekcjach powyżej. Możliwość analizy danych bez parametrów eliminuje błędy powiązane z ich niepoprawnymi wartościami. Kolejną zaletą tego typu rozwiązań jest zwiększenie wydajności obróbki danych podanych do aplikacji. Użycie tego sposobu wymaga od użytkownika podania już scharakteryzowanych danych, w wypadku wykrywania fiksacji jest to, czy dany punkt należy do fiksacji, czy jest sakadą.\par
Opis działania algorytmu, jak również sposób analizy danych został zaprezentowany w sekcji \ref{ssec:machinelearningalg}.
\section{Metody prezentacji danych}
\label{sec:othereyetracking}
Analiza fiksacji jest tylko jednym z wielu rodzajów analizy danych pochodzących z urządzenia mierzącego. Zgodnie z pracą \cite[rozdział 2.2]{OtherEyetrack}, możemy analizować dane w sposób wizualny, poprzez mapy cieplne, których przykład pokazano na rysunku \ref{fig:heatmap}. Mapy cieplne to obszary reprezentujące zwiększoną aktywność oka w danym miejscu, czy też punkty skupienia.

Podobną metodą analizy danych są ścieżki skanowania, których zadaniem jest pokazanie linii, w jakich przebiega obserwacja na obrazie. Rysunek \ref{fig:scanpaths} ukazuje przykład ścieżek skanowania wraz z numeracją, jak przebiegał ruch oka. Powyżej zaprezentowane metody posiadają jednak pewne wady, mapy cieplne ukazują nam tylko pewną reprezentacje skupienia objektu, nie ukazując nam kolejności w jakiej obraz został przeglądany. Natomiast ścieżki skanowania dają nam dokładny przebieg w jaki sposób oko obserwowało dany rysunek, jednak główną wadą, jak zauważono na rysunku \ref{fig:scanpaths} jest zbytnie nagromadzenie danych, powodowane przez to, że należy nanieść na konkretny rysunek wszystkie dane w odpowiedniej kolejności.
\begin{figure}[H]
    \centering
    \captionsetup{justification=centering,margin=2cm}
    \begin{subfigure}{.5\textwidth}
      \centering
      \includegraphics[width=\linewidth]{resources/heatmaps.png}
      \caption{Przykład mapy cieplnej}
      \label{fig:heatmap}
    \end{subfigure}%
    \begin{subfigure}{.5\textwidth}
      \centering
      \includegraphics[width=\linewidth]{resources/scanpaths.png}
      \caption{Przykład ścieżki skanowania}
      \label{fig:scanpaths}
    \end{subfigure}
    \caption[Inne reprezentacje mierzonych punktów]{Inne reprezentacje mierzonych punktów.\\\hspace{\textwidth}
    \small(źródło: \url{https://bit.ly/2MZnIev}\footnote{ze względu na długi adres, należało skrócić link} [dostęp 10.09.2019])}
    \label{fig:otherfigures}
\end{figure}